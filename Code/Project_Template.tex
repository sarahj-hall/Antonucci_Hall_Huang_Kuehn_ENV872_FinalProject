% Options for packages loaded elsewhere
\PassOptionsToPackage{unicode}{hyperref}
\PassOptionsToPackage{hyphens}{url}
\documentclass[
]{article}
\usepackage{xcolor}
\usepackage[margin=1in]{geometry}
\usepackage{amsmath,amssymb}
\setcounter{secnumdepth}{5}
\usepackage{iftex}
\ifPDFTeX
  \usepackage[T1]{fontenc}
  \usepackage[utf8]{inputenc}
  \usepackage{textcomp} % provide euro and other symbols
\else % if luatex or xetex
  \usepackage{unicode-math} % this also loads fontspec
  \defaultfontfeatures{Scale=MatchLowercase}
  \defaultfontfeatures[\rmfamily]{Ligatures=TeX,Scale=1}
\fi
\usepackage{lmodern}
\ifPDFTeX\else
  % xetex/luatex font selection
\fi
% Use upquote if available, for straight quotes in verbatim environments
\IfFileExists{upquote.sty}{\usepackage{upquote}}{}
\IfFileExists{microtype.sty}{% use microtype if available
  \usepackage[]{microtype}
  \UseMicrotypeSet[protrusion]{basicmath} % disable protrusion for tt fonts
}{}
\makeatletter
\@ifundefined{KOMAClassName}{% if non-KOMA class
  \IfFileExists{parskip.sty}{%
    \usepackage{parskip}
  }{% else
    \setlength{\parindent}{0pt}
    \setlength{\parskip}{6pt plus 2pt minus 1pt}}
}{% if KOMA class
  \KOMAoptions{parskip=half}}
\makeatother
\usepackage{longtable,booktabs,array}
\usepackage{calc} % for calculating minipage widths
% Correct order of tables after \paragraph or \subparagraph
\usepackage{etoolbox}
\makeatletter
\patchcmd\longtable{\par}{\if@noskipsec\mbox{}\fi\par}{}{}
\makeatother
% Allow footnotes in longtable head/foot
\IfFileExists{footnotehyper.sty}{\usepackage{footnotehyper}}{\usepackage{footnote}}
\makesavenoteenv{longtable}
\usepackage{graphicx}
\makeatletter
\newsavebox\pandoc@box
\newcommand*\pandocbounded[1]{% scales image to fit in text height/width
  \sbox\pandoc@box{#1}%
  \Gscale@div\@tempa{\textheight}{\dimexpr\ht\pandoc@box+\dp\pandoc@box\relax}%
  \Gscale@div\@tempb{\linewidth}{\wd\pandoc@box}%
  \ifdim\@tempb\p@<\@tempa\p@\let\@tempa\@tempb\fi% select the smaller of both
  \ifdim\@tempa\p@<\p@\scalebox{\@tempa}{\usebox\pandoc@box}%
  \else\usebox{\pandoc@box}%
  \fi%
}
% Set default figure placement to htbp
\def\fps@figure{htbp}
\makeatother
\setlength{\emergencystretch}{3em} % prevent overfull lines
\providecommand{\tightlist}{%
  \setlength{\itemsep}{0pt}\setlength{\parskip}{0pt}}
\usepackage{booktabs}
\usepackage{longtable}
\usepackage{array}
\usepackage{multirow}
\usepackage{wrapfig}
\usepackage{float}
\usepackage{colortbl}
\usepackage{pdflscape}
\usepackage{tabu}
\usepackage{threeparttable}
\usepackage{threeparttablex}
\usepackage[normalem]{ulem}
\usepackage{makecell}
\usepackage{xcolor}
\usepackage{bookmark}
\IfFileExists{xurl.sty}{\usepackage{xurl}}{} % add URL line breaks if available
\urlstyle{same}
\hypersetup{
  pdftitle={Power Outage Trends in California, Florida, Pennsylvania, and Texas from 2015 to 2023},
  hidelinks,
  pdfcreator={LaTeX via pandoc}}

\title{Power Outage Trends in California, Florida, Pennsylvania, and Texas from 2015 to 2023}
\usepackage{etoolbox}
\makeatletter
\providecommand{\subtitle}[1]{% add subtitle to \maketitle
  \apptocmd{\@title}{\par {\large #1 \par}}{}{}
}
\makeatother
\subtitle{\url{https://github.com/sarahj-hall/Antonucci_Hall_Huang_Kuehn_ENV872_FinalProject.git}}
\author{}
\date{\vspace{-2.5em}December 12th, 2025}

\begin{document}
\maketitle

{
\setcounter{tocdepth}{3}
\tableofcontents
}
\newpage
\listoftables 
\newpage
\listoffigures 
\newpage

\section{Rationale and Research Questions}\label{rationale-and-research-questions}

Electric power outages are costly for both utilities and their customers. Outages disrupt economic activity, and impact critical facilities such as hosptials (citation). The U.S. electric grid is increasingly vulnerable to outages due to increasing extreme weather events, such as hurricanes, heatwaves, wildfires, and winter storms, which are increasing in frequency and intensity due to climate change (citation). Because many of these events occur during specific times of the year, understanding weather power outages follow a seasonal pattern is important for planning and resilience. If outages tend to occur in specific months, utilities and communities can better allocate resources, resilience investments, and emergency preparedness.

As the weather events are increasing, electric utilities have also made advancements in system planning, grid hardening, and outage detection. Together this raises key questions about weather outages are becoming more or less frequent, whether they show predictable seasonal cycles, and whether their impacts are increasing over time.

To address these questions, this study focuses on Texas, California, Florida, and Pennsylvania. These states represent diverse geographic regions, climates, and electric power systems. For each state, the study examines trends in frequency of power outages, seasonality, and severity of outages. Customer weighted hours of outages will be used to quantify the impact of outages, as it combines the number of customers affected with the outage duration.

The research questions of this study are:

\begin{itemize}
\tightlist
\item
  Question 1: How has the frequency of power outages changed over time?
\item
  Question 2: Is there a seasonal trend? Are certain months more prone to outages?
\item
  Question 3: How has the severity of power outages changed over time?
\end{itemize}

\newpage

\section{Dataset Information}\label{dataset-information}

The Event-correlated Outage Dataset in America by the Pacific Northwest National Laboratory was downloaded from the Open Energy Data Initiative (OEDI) (\url{https://data.openei.org/submissions/6458}). The dataset includes an aggregated and event-correlated analysis of power outages in the United States. The specific dataset selected for this analysis is the Aggregated Outage Data which integrates data from the Environment for the Analysis of Geo-Located Energy Information (EAGLE-I), and Annual Estimates of the Resident Population for Counties 2024 (CO-EST2024-POP). The EAGLE-I dataset, provides county-level electricity outage estimates at 15-minute intervals from 2014 to 2023. It encompasses over 146 million customers, but this coverage has increased over time from 137 million in 2018. EAGLE-I only started providing data quality estimates starting in 2018. The Aggregated Outage Dataset provides monthly outage data at the state level, including total number of outages, the maximum duration of outages, and the customer weighted average of outages.

The data was wrangled by combining the yearly data from 2015 to 2023 into one dataframe. The year 2014 was removed from the analysis because it did not have monthly data, only the yearly summary. From this file, four datasets were created by filtering for each state (CA, FL, PA, and TX). For each state, the monthly value equal to 0 was filtered out, which represented the yearly summary. Additionally, a date column was added that combined the monthly and yearly columns into a date object.

This data structure shown in Table \ref{tab:datastructuretable} applies to all four state datasets. Table \ref{tab:datasummary} summarizes the key statistical characteristics of each datasets.

\begin{table}[!h]
\centering
\caption{\label{tab:datastructuretable}Summary of Outage Dataset Structure}
\centering
\begin{tabular}[t]{lll}
\toprule
Variable & Description & Units\\
\midrule
state & Two-letter state abbreviation & N/A\\
year & Year of outage & N/A\\
month & Month of outage & N/A\\
outage\_count & Number of outages per month & N/A\\
max\_outage\_duration & Longest outage duration in a month & Hours\\
\addlinespace
customer\_weighted\_hours & Customer-weighted outage hours & N/A\\
date & Date of outage & N/A\\
\bottomrule
\end{tabular}
\end{table}

\begin{table}[!h]
\centering
\caption{\label{tab:datasummary}Summary Statisitcs of CA, FL, PA, and TX Datasets}
\centering
\resizebox{\ifdim\width>\linewidth\linewidth\else\width\fi}{!}{
\begin{tabular}[t]{ccccccc}
\toprule
State & \makecell[c]{Outage Count\\Range} & \makecell[c]{Outage Count\\Mean} & \makecell[c]{Max Duration\\Range} & \makecell[c]{Max Duration\\Mean} & \makecell[c]{Customer Weighted\\Hours Range} & \makecell[c]{Customer Weighted\\Hours Mean}\\
\midrule
CA & 140–2599 & 900.55 & 27.25–609.75 & 131.82 & 491034.75–88045909.75 & 6,151,189\\
FL & 402–1515 & 970.73 & 11.5–532.75 & 56.80 & 438767.25–532294703.25 & 9,390,167\\
PA & 336–1312 & 600.20 & 13.5–140.5 & 46.74 & 333846–28965544.5 & 1,970,894\\
TX & 120–2361 & 1,108.82 & 6–723.75 & 76.26 & 167452–227923742 & 5,712,808\\
\bottomrule
\end{tabular}}
\end{table}

\newpage

\section{Exploratory Analysis}\label{exploratory-analysis}

Insert exploratory visualizations of your dataset. This may include, but is not limited to, graphs illustrating the distributions of variables of interest and/or maps of the spatial context of your dataset. Format your R chunks so that graphs are displayed but code is not displayed. Accompany these graphs with text sections that describe the visualizations and provide context for further analyses.

Each figure should be accompanied by a caption, and each figure should be referenced within the text.

Scope: think about what information someone might want to know about the dataset before analyzing it statistically. How might you visualize this information?

\subsubsection{California}\label{california}

Initial data exploration of California power outage data suggest a slight increasing trend (Figure \ref{fig:cayearly}). Furthermore, Figure \ref{fig:camonthly} shows.

\begin{figure}

{\centering \includegraphics{Project_Template_files/figure-latex/cayearly-1} 

}

\caption{Yearly plot of power outages in California from 2015 to 2023.}\label{fig:cayearly}
\end{figure}

\begin{figure}

{\centering \includegraphics{Project_Template_files/figure-latex/camonthly-1} 

}

\caption{Monthly plot of power outages in California from 2015 to 2023.}\label{fig:camonthly}
\end{figure}

\newpage

\subsubsection{Florida}\label{florida}

\begin{figure}

{\centering \includegraphics{Project_Template_files/figure-latex/flyearly-1} 

}

\caption{Yearly plot of power outages in Florida from 2015 to 2023.}\label{fig:flyearly}
\end{figure}

\begin{figure}

{\centering \includegraphics{Project_Template_files/figure-latex/flmonthly-1} 

}

\caption{Monthly plot of power outages in Florida from 2015 to 2023.}\label{fig:flmonthly}
\end{figure}

\newpage

\subsubsection{Pennsylvania}\label{pennsylvania}

From the initial data exploration of Pennsylvania, it is clear that power outages have gradually increased over the past decade. On a monthly basis, there is a noticeable rise in outages during the summer months, suggesting a seasonal pattern likely related to weather or energy demand. Overall, the data points to both long-term growth in outage frequency and predictable seasonal fluctuations.

\begin{figure}

{\centering \includegraphics{Project_Template_files/figure-latex/payearly-1} 

}

\caption{Yearly plot of power outages in Pennsylvania from 2015 to 2023.}\label{fig:payearly}
\end{figure}

\begin{figure}

{\centering \includegraphics{Project_Template_files/figure-latex/pamonthly-1} 

}

\caption{Monthly plot of power outages in Pennsylvania from 2015-2023.}\label{fig:pamonthly}
\end{figure}

\newpage

\subsubsection{Texas}\label{texas}

\begin{figure}

{\centering \includegraphics{Project_Template_files/figure-latex/txyearly-1} 

}

\caption{Yearly plot of power outages in Texas from 2015-2023.}\label{fig:txyearly}
\end{figure}

\begin{figure}

{\centering \includegraphics{Project_Template_files/figure-latex/txmonthly-1} 

}

\caption{Monthly plot of power outages in Texas from 2015 to 2023.}\label{fig:txmonthly}
\end{figure}

\newpage

\section{Analysis}\label{analysis}

Insert visualizations and text describing your main analyses. Format your R chunks so that graphs are displayed but code and other output is not displayed. Instead, describe the results of any statistical tests in the main text (e.g., ``Variable x was significantly different among y groups (ANOVA; df = 300, F = 5.55, p \textless{} 0.0001)''). Each paragraph, accompanied by one or more visualizations, should describe the major findings and how they relate to the question and hypotheses. Divide this section into subsections, one for each research question.

Each figure should be accompanied by a caption, and each figure should be referenced within the text

**trying to figure out the best organization for this.. where is the best place for the seasonality component?

\subsection{Question 1: How has the frequency of power outages changed over time?}\label{question-1-how-has-the-frequency-of-power-outages-changed-over-time}

\subsubsection{California}\label{california-1}

Figure \ref{fig:CAtimeseries} displays the decomposed time series of the California power outage frequency from 2015 to 2023. There appears to be clear seasonal trend, which is analyzed further in part (5).

A MannKendall non seasonal trend analysis was applied to the California power outage frequency dataset with the seasonal component remove, a significant overall increase in power outages over time is observed (MannKendall; tau = 5.437e-01 p \textless{} 2.2e-16). Figure \ref{fig:CAtrend} visualizes the trend compared to the observed frequency of power outages in California. The red line represents the overall increasing trend.

\begin{figure}

{\centering \includegraphics{Project_Template_files/figure-latex/CAtimeseries-1} 

}

\caption{Decomposed components of the California power outage count time series.}\label{fig:CAtimeseries}
\end{figure}

\begin{figure}

{\centering \includegraphics{Project_Template_files/figure-latex/CAtrend-1} 

}

\caption{Trend versus observation of power outages in California (2015-2023).}\label{fig:CAtrend}
\end{figure}

\newpage

\subsubsection{Florida}\label{florida-1}

\begin{figure}

{\centering \includegraphics{Project_Template_files/figure-latex/FLtimeseries-1} 

}

\caption{Time series for FL monthly power outages from 2015 to 2023.}\label{fig:FLtimeseries}
\end{figure}

\begin{figure}

{\centering \includegraphics{Project_Template_files/figure-latex/FLdecomp-1} 

}

\caption{FL Trend Analysis for Monthly Outages}\label{fig:FLdecomp}
\end{figure}

\newpage

\subsubsection{Pennsylvania}\label{pennsylvania-1}

After removing the seasonal component from the Pennsylvania outage time series, the Mann--Kendall trend test revealed a significant positive trend (z = 3.02, n = 108, p = 0.002), indicating that outage frequency has increased from 2015 to 2023. This result supports the hypothesis that outages have become more common during this time, independent of seasonal patterns. The decomposed time-series visualization shows this trend clearly, revealing that even after accounting for monthly fluctuations, the long-term component continues to rise.

\begin{figure}

{\centering \includegraphics{Project_Template_files/figure-latex/PAtimeseries-1} 

}

\caption{Decomposed components of the Pennsylvania power outage count time series.}\label{fig:PAtimeseries}
\end{figure}

\begin{figure}

{\centering \includegraphics{Project_Template_files/figure-latex/PAtrend-1} 

}

\caption{Trend versus observation of power outages in Pennsylvania (2015-2023).}\label{fig:PAtrend}
\end{figure}

\newpage

\subsubsection{Texas}\label{texas-1}

\begin{figure}

{\centering \includegraphics{Project_Template_files/figure-latex/txtimeseries-1} 

}

\caption{Decomposed components of the Texas power outage count time series.}\label{fig:txtimeseries}
\end{figure}

\begin{figure}

{\centering \includegraphics{Project_Template_files/figure-latex/TXtrend-1} 

}

\caption{Trend versus observation of power outages in Texas (2015-2023).}\label{fig:TXtrend}
\end{figure}

\newpage

\subsubsection{Comparision between States}\label{comparision-between-states}

\newpage

\subsection{Question 2: Is there a seasonal trend? Are certain months more prone to outages?}\label{question-2-is-there-a-seasonal-trend-are-certain-months-more-prone-to-outages}

\subsubsection{California}\label{california-2}

Figure \ref{fig:CAtimeseries} displays a clear seasonal trend in the decomposed time series of the California power outage frequency. The seasonal trend was isolated and grouped by year to show the Seasonal Cycle of Power Outages in California in Figure \ref{fig:CAseasonal}. The seasonal cycle highlights a potential pattern of power outage frequencies peaking rising in the summer months, peaking in October. However, the frequency of outages in California was not significantly different among the months (ANOVA; df = 11, F = 0.704, p \textless{} 0.732).

\begin{figure}

{\centering \includegraphics{Project_Template_files/figure-latex/CAseasonal-1} 

}

\caption{Seasonality component of the California power outage count time series analysis where yearly data is grouped by month.}\label{fig:CAseasonal}
\end{figure}

\newpage

\subsubsection{Florida}\label{florida-2}

\newpage

\subsubsection{Pennsylvania}\label{pennsylvania-2}

The Mann--Kendall seasonal trend test for Pennsylvania outage counts indicated a significant seasonal signal in the data (\(\tau = 0.20\), p = 0.009), demonstrating that outage frequencies vary across months. This finding supports the hypothesis that outages are not evenly distributed throughout the year but instead follow a seasonal pattern. The visualizations further highlight this pattern, showing clear peaks in June, July, and August, when outage counts are highest. Together, these results suggest that summer months consistently experience elevated outage activity, potentially driven by seasonal weather conditions or increased energy demand.

\begin{figure}

{\centering \includegraphics{Project_Template_files/figure-latex/unnamed-chunk-2-1} 

}

\caption{Isolated seasonal component of PA power outages.}\label{fig:unnamed-chunk-2}
\end{figure}

\begin{figure}

{\centering \includegraphics{Project_Template_files/figure-latex/PAseasonal-1} 

}

\caption{Single year breakdown of seasonal component of outages in PA.}\label{fig:PAseasonal}
\end{figure}

\newpage

\subsubsection{Texas}\label{texas-2}

\begin{figure}

{\centering \includegraphics{Project_Template_files/figure-latex/TXseasonal-1} 

}

\caption{Seasonality component of the California power outage count time series analysis where yearly data is grouped by month.}\label{fig:TXseasonal}
\end{figure}

\newpage

\subsubsection{Comparision between States}\label{comparision-between-states-1}

\newpage

\subsection{Question 3: How has the length of power outages changed over time?}\label{question-3-how-has-the-length-of-power-outages-changed-over-time}

\subsubsection{California}\label{california-3}

Figure \ref{fig:CAcwh} displays the decomposed time series of the California customer weighted hours from 2015 to 2023. However, after removing the seasonality component the Mann-Kendall trend test suggested that there is a significant increase in the customer weighted hours of power outages in California, thus the outage severity (MannKendall; tau 3.513e-01 = p \textless{} 7.178e-08).

\begin{figure}

{\centering \includegraphics{Project_Template_files/figure-latex/CAcwh-1} 

}

\caption{Decomposed analyis of customer weighted outage time in California}\label{fig:CAcwh}
\end{figure}

\newpage

\subsubsection{Florida}\label{florida-3}

\newpage

\subsubsection{Pennsylvania}\label{pennsylvania-3}

For customer-weighted hours, the seasonal test was not statistically significant (tau = 0.13, p = 0.09), indicating no consistent monthly pattern in outage impacts on customers. However, after de-seasonalizing, the Mann-Kendall trend test suggested a marginally increasing trend in customer-weighted hours over time (z = 1.89, n = 108, p = 0.06). Overall, these results suggest that while outage counts are increasing and show clear seasonality, the impact on customers is also trending upward, though less strongly.

\begin{figure}

{\centering \includegraphics{Project_Template_files/figure-latex/PAcwh-1} 

}

\caption{Decomposed analyis of customer weighted outage time in Pennsylvania.}\label{fig:PAcwh}
\end{figure}

\begin{figure}

{\centering \includegraphics{Project_Template_files/figure-latex/PAcwhTrend-1} 

}

\caption{Trend versus observation of customer-weighted hours in PA (2015-2023).}\label{fig:PAcwhTrend}
\end{figure}

\begin{figure}

{\centering \includegraphics{Project_Template_files/figure-latex/PAcwhSeas-1} 

}

\caption{Single year breakdown of seasonal component of customer-weighted hours in PA.}\label{fig:PAcwhSeas}
\end{figure}

\newpage

\subsubsection{Texas}\label{texas-3}

\newpage

\subsubsection{Comparison between States}\label{comparison-between-states}

\newpage

\section{Summary and Conclusions}\label{summary-and-conclusions}

Summarize your major findings from your analyses in a few paragraphs. What conclusions do you draw from your findings? Relate your findings back to the original research questions and rationale.

After accounting for seasonal patterns, the analysis shows that the frequency of power outages in Pennsylvania has increased from 2015 to 2023. Outage counts display a clear seasonal trend, with the highest numbers occurring in the summer months of June, July, and August. In contrast, customer-weighted hours do not show a strong seasonal pattern, but there is a slight upward trend over time, indicating that the impact of outages on customers is also gradually increasing. Overall, while outages are becoming more frequent and seasonally concentrated, the effect on customers is rising more moderately.

\newpage

\section{References}\label{references}

\textless add references here if relevant, otherwise delete this section\textgreater{}

\end{document}
